\documentclass{article}
\oddsidemargin 0in
\textwidth 6.75in 
\topmargin 0in
\textheight 8.5in
\parindent 0em
\parskip 2ex

\begin{document}

\begin{center}
    \Large{\textbf{Programación 1er semestre y proyecciones 2do semestre}}
\end{center}

\section*{1. Introducción}

\subsection*{1.1 ¿En qué consistirá este documento?}
Este documento consiste en una recapitulación de los conocimientos adquiridos a lo largo del primer semestre sobre lógica de programación y aplicación en Python, métodos de estudio, pensamientos personales, planteamiento de próximo proyecto y las respectivas herramientas para su realización.

\subsection*{1.2 ¿Cómo te preparaste para este curso?}
A finales del año anterior, luego de tomar la decisión de unirme al diferenciado de programación, comencé con un pequeño curso de las bases de programación en C++ y diagramas de flujo, ya que quería conocer un poco sobre lo que haríamos en el diferenciado, sin embargo, al comenzar el año se nos presentó Python como el lenguaje que utilizaríamos para este curso lo que me confundió un poco, pero luego de unas clases introductorias logré entender lo más básico.

\section*{2. Modo de aprendizaje en programación}

\subsection*{2.1 ¿Tenías alguna experiencia previa en la programación?}
Lo único que realmente conocía con anterioridad era la lógica proposicional, la cual me ayudó bastante a comenzar a comprender la lógica de programación, pero en demás no tenía mucha idea de cómo me desarrollaría en el tema.

\subsection*{2.2 ¿Qué recursos utilizas para aprender programación?}
Los métodos que utilice para poder llegar a donde estoy fueron, aparte de las clases principales, hice el curso de Coursera de la Universidad Católica, vi bastantes videos interesantes, comencé distintos cursos de Domestika y Udemy (no terminé varios, pero todos aportaron nuevos conocimientos), también practiqué en casa con pequeños códigos los que me generaron estrés, pero al lograr completarlos todo era satisfacción y obviamente chat GPT me ayudó mucho a resolver dudas al igual que el profesor que está siempre abierto a ayudarte a progresar, que en mi caso fue de mucha utilidad y estoy agradecida con él.

\subsection*{2.2 ¿Qué habilidades adquiriste durante el proceso de aprendizaje?}
Las habilidades que adquirí durante estos meses son variadas, comenzando con los comandos básicos de Python, luego con lógica de programación, tipos de variable, definición de funciones, trabajar con distintas librerías como tkinter o pygame para realizar una interfaz gráfica, programación orientada a objetos y finalmente cómo usar la plataforma de GIT y sus comandos básicos.

\section*{3. Progreso en programación}

\subsection*{3.1 ¿Qué logros alcanzaste o crees haber alcanzado en estos meses?}
En lo personal creo haber avanzado bastante ya que empecé desde una base casi completamente nula, a poder crear un programa (no demasiado avanzado) sin necesidad de consultar a internet, también puedo manejar una plataforma como lo es GIT, de la cual no tenía idea antes de estar en este curso, además que cuando comento lo que estoy haciendo con personas de mi alrededor me dicen que es algo muy complicado o que no podrían llegar a entender y muchas de las veces siento lo mucho que e avanzado porque siento que lo que estoy haciendo es bastante fácil y podría seguir desafiandome.

\subsection*{3.2 ¿Qué proyectos realizaste este semestre?}
En lo personal creo que lo que hice este semestre es bueno, pero puede ser mucho mejor, hice un programa que daba ejercicios matemáticos en un rango dado por el usuario, también le di interfaz gráfica y agregué sonidos para las repuestas del usuario, sin embargo,  constantemente recibí comentarios por parte de gente cercana diciendo que lo que hacía no tenía sentido, que era inútil y que no me servía de nada ocupar tanto tiempo de mi día trabajando en eso, lo que provocó un poco de desmotivación en mí, pero luego de un tiempo decidí continuar y no tomar en cuenta los comentarios, esperando poder probar lo contrario y que apoyen mis decisiones.

\section*{4. Proyecto futuro}

\subsection*{4.1 ¿Cuál es la idea de proyecto que tienes?}
Me gustaría crear una aplicación de teléfono que ayude con la despensa de una cocina, si bien no es lo que tenía pensado desde un principio, es algo que ayudaría mucho a mi mamá que últimamente se está dedicando un poco a esto.

\subsection*{4.2 ¿Cuál es el objetivo de tu proyecto?}
El objetivo de esta aplicación es que tenga un control de elementos que hay en nuestra cocina ingresados por el usuario con su fecha de vencimiento y que la aplicación indique cuánto falta para que la comida caduque además de dar distintos consejos sobre qué   hacer o que cocinar con ciertos ingredientes. Esta aplicación también tendría registro de usuario y contraseña.

\section*{5. Especificaciones del proyecto}

\subsection*{5.1 Tecnologías, lenguajes de programación y herramientas a utilizar.}

\subsection*{5.2 Requisitos funcionales y no funcionales del proyecto.}

\subsection*{5.3 Alcance y limitaciones del proyecto.}

\section*{6. Conclusiones}

\subsection*{6.1 Recapitulación del modo de aprendizaje en programación y progreso.}

\subsection*{6.2 Expectativas y metas personales en relación al proyecto futuro.}


\end{document}